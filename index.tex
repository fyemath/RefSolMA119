% Options for packages loaded elsewhere
\PassOptionsToPackage{unicode}{hyperref}
\PassOptionsToPackage{hyphens}{url}
\PassOptionsToPackage{dvipsnames,svgnames,x11names}{xcolor}
%
\documentclass[
  12pt]{article}

\usepackage{amsmath,amssymb}
\usepackage[]{libertinus}
\usepackage{iftex}
\ifPDFTeX
  \usepackage[T1]{fontenc}
  \usepackage[utf8]{inputenc}
  \usepackage{textcomp} % provide euro and other symbols
\else % if luatex or xetex
  \usepackage{unicode-math}
  \defaultfontfeatures{Scale=MatchLowercase}
  \defaultfontfeatures[\rmfamily]{Ligatures=TeX,Scale=1}
\fi
% Use upquote if available, for straight quotes in verbatim environments
\IfFileExists{upquote.sty}{\usepackage{upquote}}{}
\IfFileExists{microtype.sty}{% use microtype if available
  \usepackage[]{microtype}
  \UseMicrotypeSet[protrusion]{basicmath} % disable protrusion for tt fonts
}{}
\makeatletter
\@ifundefined{KOMAClassName}{% if non-KOMA class
  \IfFileExists{parskip.sty}{%
    \usepackage{parskip}
  }{% else
    \setlength{\parindent}{0pt}
    \setlength{\parskip}{6pt plus 2pt minus 1pt}}
}{% if KOMA class
  \KOMAoptions{parskip=half}}
\makeatother
\usepackage{xcolor}
\usepackage[margin=0.8in]{geometry}
\setlength{\emergencystretch}{3em} % prevent overfull lines
\setcounter{secnumdepth}{-\maxdimen} % remove section numbering
% Make \paragraph and \subparagraph free-standing
\ifx\paragraph\undefined\else
  \let\oldparagraph\paragraph
  \renewcommand{\paragraph}[1]{\oldparagraph{#1}\mbox{}}
\fi
\ifx\subparagraph\undefined\else
  \let\oldsubparagraph\subparagraph
  \renewcommand{\subparagraph}[1]{\oldsubparagraph{#1}\mbox{}}
\fi


\providecommand{\tightlist}{%
  \setlength{\itemsep}{0pt}\setlength{\parskip}{0pt}}\usepackage{longtable,booktabs,array}
\usepackage{calc} % for calculating minipage widths
% Correct order of tables after \paragraph or \subparagraph
\usepackage{etoolbox}
\makeatletter
\patchcmd\longtable{\par}{\if@noskipsec\mbox{}\fi\par}{}{}
\makeatother
% Allow footnotes in longtable head/foot
\IfFileExists{footnotehyper.sty}{\usepackage{footnotehyper}}{\usepackage{footnote}}
\makesavenoteenv{longtable}
\usepackage{graphicx}
\makeatletter
\def\maxwidth{\ifdim\Gin@nat@width>\linewidth\linewidth\else\Gin@nat@width\fi}
\def\maxheight{\ifdim\Gin@nat@height>\textheight\textheight\else\Gin@nat@height\fi}
\makeatother
% Scale images if necessary, so that they will not overflow the page
% margins by default, and it is still possible to overwrite the defaults
% using explicit options in \includegraphics[width, height, ...]{}
\setkeys{Gin}{width=\maxwidth,height=\maxheight,keepaspectratio}
% Set default figure placement to htbp
\makeatletter
\def\fps@figure{htbp}
\makeatother

\usepackage{fancyhdr}
\fancyhf{}
\pagestyle{fancy}
\lhead{Spring 2022}
\rhead{Reuse: \href{https://creativecommons.org/licenses/by-nc-sa/3.0/}{CC BY-NC-SA 3.0}}
\makeatletter
\makeatother
\makeatletter
\@ifpackageloaded{caption}{}{\usepackage{caption}}
\AtBeginDocument{%
\ifdefined\contentsname
  \renewcommand*\contentsname{Table of contents}
\else
  \newcommand\contentsname{Table of contents}
\fi
\ifdefined\listfigurename
  \renewcommand*\listfigurename{List of Figures}
\else
  \newcommand\listfigurename{List of Figures}
\fi
\ifdefined\listtablename
  \renewcommand*\listtablename{List of Tables}
\else
  \newcommand\listtablename{List of Tables}
\fi
\ifdefined\figurename
  \renewcommand*\figurename{Figure}
\else
  \newcommand\figurename{Figure}
\fi
\ifdefined\tablename
  \renewcommand*\tablename{Table}
\else
  \newcommand\tablename{Table}
\fi
}
\@ifpackageloaded{float}{}{\usepackage{float}}
\floatstyle{ruled}
\@ifundefined{c@chapter}{\newfloat{codelisting}{h}{lop}}{\newfloat{codelisting}{h}{lop}[chapter]}
\floatname{codelisting}{Listing}
\newcommand*\listoflistings{\listof{codelisting}{List of Listings}}
\makeatother
\makeatletter
\@ifpackageloaded{caption}{}{\usepackage{caption}}
\@ifpackageloaded{subcaption}{}{\usepackage{subcaption}}
\makeatother
\makeatletter
\@ifpackageloaded{tcolorbox}{}{\usepackage[many]{tcolorbox}}
\makeatother
\makeatletter
\@ifundefined{shadecolor}{\definecolor{shadecolor}{rgb}{.97, .97, .97}}
\makeatother
\makeatletter
\makeatother
\ifLuaTeX
  \usepackage{selnolig}  % disable illegal ligatures
\fi
\IfFileExists{bookmark.sty}{\usepackage{bookmark}}{\usepackage{hyperref}}
\IfFileExists{xurl.sty}{\usepackage{xurl}}{} % add URL line breaks if available
\urlstyle{same} % disable monospaced font for URLs
\hypersetup{
  pdftitle={Reference Solutions for MA119 Final Review},
  pdfauthor={Dr.~Ye},
  colorlinks=true,
  linkcolor={blue},
  filecolor={Maroon},
  citecolor={Blue},
  urlcolor={Blue},
  pdfcreator={LaTeX via pandoc}}

\title{Reference Solutions for MA119 Final Review}
\author{Dr.~Ye}
\date{}

\begin{document}
\maketitle
\ifdefined\Shaded\renewenvironment{Shaded}{\begin{tcolorbox}[boxrule=0pt, interior hidden, breakable, enhanced, sharp corners, borderline west={3pt}{0pt}{shadecolor}, frame hidden]}{\end{tcolorbox}}\fi

\begin{enumerate}
\def\labelenumi{\arabic{enumi}.}
\item
  Applying negative exponent, product, quotient-to-power,
  product-to-power, and power-to-power rules implies \[
   \begin{aligned}
   \left(\frac{6x^5y^{-2}z}{2x^{-3}y^4z^{-11}}\right)^{-3}=&\left(\frac{3x^5x^3zz^{11}}{y^4y^2}\right)^{-3}\\
   =&\left(\frac{3x^8z^{12}}{y^6}\right)^{-3}\\
   =&\left(\frac{y^6}{3x^8z^{12}}\right)^3\\
   =&\frac{(y^6)^3}{3^3(x^8)^3(z^{12})^3}\\
   =&\frac{y^{18}}{27x^{24}z^{36}}\\
   \end{aligned}
   \]
\item
  Solving for \(y\) yields \(y=\frac12x-\frac74\). So the slope of the
  given line is \(m=\frac12\). Then the slope of the perpendicular line
  is \(m_\perp=-\frac1m=-2\). Since the perpendicular line passes
  through \((2,-5)\), it is determined by the equation

  \[y=-2(x-2)-5.\]

  Simplifying the right hand side yields the slope-intersect form

  \[y=-2x-1.\]
\item
  The slope of the given line is

  \[m=\frac{4-6}{-2-2}=\frac{-2}{-4}=\frac12.\]

  The point-slope form equation of the line is

  \[y=\frac12(x-2)+6.\]

  Simplifying the right hand side gives the slope-intercept form
  equation

  \[y=\frac12x+5.\]
\item
  The first step to solve an absolut value equation is to isolate the
  absolute value expression. Then using the definition of absolute value
  to remove the absolute value sign and solving the resulting equation.
  It's better to check the answers. In this case, \[
   \begin{aligned}
   |3x+2|+1=&6\\
   |3x+2|=&5\\
   3x+2=5\quad\text{or}&\quad 3x+2=-5\\
   3x=3\quad\text{or}&\quad 3x=-7\\
   x=1\quad\text{or}&\quad x=-\frac73.
   \end{aligned}
   \]

  \textbf{Check:} when \(x=1\), the left hand side is
  \(|3\cdot 1 + 2|+1=|5|+1=5+1=6\) which equals the right hand side. So
  \(x=1\) is a solution. When \(x=-\frac73\), the left hand side is
  \(|3\cdot(-\frac73)+3|+1=|-5|+1=5+1=6\) which equals the right hand
  side. So \(x=-\frac73\) is also a solution.
\item
  When using grouping method to factor polynomial with four terms, you
  can expect the second group has the same binomial factor.

  \begin{enumerate}
  \def\labelenumii{\arabic{enumii}.}
  \tightlist
  \item
  \end{enumerate}

  \[
   \begin{aligned}
   12xy-10y-18x+15=&(12xy-10y)+(-18x+15)\\
   =&2y(6x-5)+(6x-5)(-3)\\
   =&(6x-5)(2y-3).
   \end{aligned}
   \]

  \begin{enumerate}
  \def\labelenumii{\arabic{enumii}.}
  \setcounter{enumii}{1}
  \tightlist
  \item
  \end{enumerate}

  \[
   \begin{aligned}
   x^3+5x^2-16x-80=&(x^3+5x^2)+(-16x-80)\\
   =&x^2(x+5)+(x+5)(-16)\\
   =&(x+5)(x^2-16)\\
   =&(x+5)(x^2-4^2)\\
   =&(x+5)(x-4)(x+4).
   \end{aligned}
   \]
\item
  The graph has two \(x\)-intercepts \((-1, 0)\) and \((3,0)\) and one
  \(y\)-intercept \((0, 3)\). The graph between the two \(x\)-intercepts
  is above the \(x\)-axis, equivalently, \(y>0\). So the \(y\)-value is
  positive if \(x\) is in \((-1,3)\). The domain of the graph is
  \((-\infty, \infty)\) because any vertical line will intersect the
  graph. The range of the graph is \((-\infty, 4]\) because it has a
  highest positioned point \((1, 4)\) but no lowest positioned point.
  The value \(f(1)=4\) as \((1, 4)\) is the point on the graph. Since
  \(f(0)=3\) and \(f(2)=3\), the equation \(f(x)=3\) has two solutions
  \(x=0\) and \(x=2\).
\item
  Solving a linear inequality is very similar to solving a linear
  equation. But for a linear inequality, when multiplying or dividing a
  negative number, the inequality sign should be switched. \[
   \begin{aligned}
   4x-5\leq& 3(x+2)\\
   4x-5\leq& 3x+6\\
   x\leq&11.
   \end{aligned}
   \] As \(x\leq 11\), \(x\) is on the left hand side of 11. That is 11
  is the right bound of the interval. There is no left bound so the
  interval for \(x\le 11\) is \((-\infty, 11]\).
\item
  When factoring a binomial involving cubes, you may need the
  sum/difference of cubes formula
  \(A^3\pm B^3=(A\pm B)(A^2\mp AB+B^2)\).

  \[x^3y-8y=y(x^3-8)=y(x-2)(x^2+2x+4).\]
\item
  One way to solve a rational equation is to first clearing the
  denominator by multiplying the LCD. However, when using this method,
  after solving the resulting equation, the solutions must be checked
  for possible extraneous solutions. In this case, the LCD is
  \((x-5)(x+1)=x^2-4x-5\). \[
   \begin{aligned}
   \frac{3x}{x-5}-\frac{2x}{x+1}=&-\frac{42}{x^2-4x-5}\\
   (x-5)(x+1)\cdot\frac{3x}{x-5}-(x-5)(x+1)\cdot\frac{2x}{x+1}=&(x-5)(x+1)\cdot\left(-\frac{42}{x^2-4x-5}\right)\\
   3x(x+1)-2x(x-5)=&-42\\
   3x^2+3x-2x^2+10x=&-42\\
   x^2+13x+42=&0\\
   (x+6)(x+7)=&0\\
   x+6=0\quad\text{or}&\quad x+7=0\\
   x=-6\quad\text{or}&\quad x=-7.
   \end{aligned}
   \] Since the LCD is nonzero for both \(x=-6\) and \(x=-7\), they are
  not extraneous solutions.
\item
  For a square root to be real, the radicand has to be nonnegative.
  Therefore, the domain of the function \(f(x)=\sqrt{-4-2x}\) is the
  solution set of the inequality \(-4-2x\ge 0\). Solving this linear
  inequality implies \(x\le -2\). In interval notation, the domain is
  \((-\infty, -2]\).
\item
  Since \(-27=(-3)^3\), \((-27)^{\frac23}=((-3)^3)^{\frac23}=(-3)^2=9\).
\item
  Rational exponents have the same rules of integer exponents. Moreover,
  \(x^{\frac mn}=\sqrt[n]{x^m}=(\sqrt[n]{x})^m\) given that
  \(\sqrt[n]x\) is real. Then
\end{enumerate}

\[
(-8x^6y^2)^{\frac13}=((-2)^3)^{\frac13}(x^6)^{\frac13}(y^2)^{\frac13}=(-2)^1x^2y^{\frac23}=-2x^2\sqrt[3]{y^2}.
\]

\begin{enumerate}
\def\labelenumi{\arabic{enumi}.}
\setcounter{enumi}{12}
\item
  Solving a radical equation is similar to solving an absolute value
  equation or an exponential equation. The radical expression should be
  isolated first. Then the functional transformation, taking power of
  both sides, and be applied. But be careful, not all functional
  transformations are equivalent transformations. \[
  \begin{aligned}
  \sqrt{x+2}+4=&x\\
  \sqrt{x+2}=&x-4\\
  (\sqrt{x+2})^2=&(x-4)^2\\
  x+2=&x^2-2\cdot 4\cdot x+4^2\\
  x^2-8x+16=&x+2\\
  x^2-9x+14=&0\\
  (x-2)(x-7)=&0\\
  x-2=0\quad\text{or}&\quad x-7=0\\
  x=2\quad\text{or}&\quad x=7.
  \end{aligned}
  \] \textbf{Check:} when \(x=2\), the left hand side is
  \(\sqrt{2+2}+4=\sqrt{4}+4=2+4=6\) which does not equal the right hand
  side. So \(x=2\) is not a solution. When \(x=7\), the left hand side
  is \(\sqrt{7+2}+4=\sqrt{9}+4=3+4=7\) which equals the right hand side.
  So \(x=7\) is a solution.
\item
  Note that \(\mathbf{i}^2=-1\) and \((A\pm B)^2=A^2\pm 2AB+B^2\). Then

  \[(3+4\mathbf{i})^2=3^2+2\cdot 3\cdot 4\mathbf{i}+4^2\mathbf{i}^2=9+24\mathbf{i}-16=-7+24\mathbf{i}.\]
\item
  Recall \(\sqrt{-b}=\mathbf{i}\sqrt{b}\) if \(b>0\). Then \[
  \begin{aligned}
  (x-7)^2=&-16\\
  x-7=&\pm\sqrt{-16}\\
  x-7=&\pm\sqrt{16}\cdot\mathbf{i}\\
  x-7=&\pm4\mathbf{i}\\
  x=&7\pm4\mathbf{i}.\\
  \end{aligned}
  \]
\item
  Suppose the width of the frame is \(x\) centimeters, then the
  dimension of the combined area is \(2x+50\) centimeters by \(2x+30\)
  centimeters and \(x\) satisfies the equation

  \[(2x+50)(2x+30)=2204.\]

  Solving this equation implies \[
  \begin{aligned}
  4x^2+160x+1500=&2204\\
  4x^2+160x-704=&0\\
  x^2+40x-176=&0\\
  (x-4)(x-44)=&0\\
  x-4=0\quad\text{ or}&\quad x+44=0\\
  x=4\quad\text{ or}&\quad x=-44.
  \end{aligned}
  \] Therefore, the width of the frame is \(4\) centimeters.
\item
  Suppose the positive number is \(x\). The sum of the number and twice
  of its square can be expressed as \(x+2x^2\). Then \(x\) satisfies the
  equation \(x+2x^2=11.88\) which can be re-written as
  \(2x^2+x-11.88=0\). Solving by the quadratic formula yields

  \[x=\frac{-1+\sqrt{1^2-4\cdot 2\cdot (-11.88)}}{2\cdot 2}=2.2.\]
\item
  In the investment formula \(A=P(1+\frac{r}{n})^{nt}\), \(P\) is the
  initial investment, \(r\) is the annual interest rate, \(n\) is the
  number of compounding in a year, \(t\) is the number of years, and
  \(A\) is the balance after \(t\) years. In this case, \(P=20000\),
  \(r=5\%=0.05\), \(n=4\) since the interest is compounded quarterly,
  that is 4 times per year. Then the balance \(A\) after 4 years is

  \[A=20000\left(1+\frac{0.05}{4}\right)^{4\cdot 4}\approx 24397.79.\]
\item
  Recall \(\log_bx\) is the number \(y\) such that \(b^y=x\), that is,
  \(b^{\log_bx}=x\). To solve the equation, taking both sides as
  exponents of the base of the log implies
\end{enumerate}

\[
\begin{aligned}
\log_8x=&\frac43\\
8^{\log_8x}=&8^{\frac43}\\
x=&(2^3)^{\frac43}\\
x=&2^4\\
x=&16.
\end{aligned}
\]

\begin{enumerate}
\def\labelenumi{\arabic{enumi}.}
\setcounter{enumi}{19}
\item
  By the power rule of log, \(\log_b(M^p)=p\log_bM\). Applying \(\log\)
  (or \(\ln\)) to a power will bring down the exponent. \[
  \begin{aligned}
  3^x=&62\\
  \log(3^x)=&\log(62)\\
  x\log(3)=&\log(62)\\
  x=&\frac{\log(62)}{\log(3)}\\
  x\approx & 3.757.
  \end{aligned}
  \]
\item
  To solve a logarithmic equation involving more than one logarithms,
  the first step is to rewrite them into a single logarithm using rules
  of logarithms. \[
  \begin{aligned}
  \log(16-20x)-\log(-x)=&2\\
  \log\left(\frac{16-20x}{-x}\right)=&2\qquad \text{quotient rule applied}\\
  \left(\frac{16-20x}{-x}\right)=&10^2\qquad \text{apply the definition}\\
  \left(\frac{20x-16}{x}\right)=&100 \qquad \text{simplify}\\
  20x-16=&100x \qquad \text{clear denominator}\\
  -80x=&16 \qquad \text{adding } -100x+16 \text{ to both sides}\\
  x=&\frac{16}{-80} \qquad \text{dividing } -80 \text{ from both sides}\\
  x=&-\frac{1}{5} \qquad \text{simplify}.
  \end{aligned}
  \] \textbf{Note:} Because the domain of a logarithmic function
  \(y=\log_bx\) is \((0, \infty)\) instead of \((-\infty,\infty)\).
  There may be extraneous solutions. After solving logarithmic function,
  you need to check if the solution makes all logarithmic expressions
  well defined. In this case, when \(x=-\frac15\), \(16-5x=20\) and
  \(-x=\frac15\). So both logarithmic expression are well defined and
  \(x=-\frac15\) is not an extraneous solution.
\item
  The notation \(f(x)\) represents the output of the function \(f\) for
  the given input \(x\). When the function is defined by an equation, to
  evaluate \(f(a)\), simply replace \(x\) by \(a\) and evaluate.

  \[f(0)=0^2-7\cdot 0 +4=4.\]

  \[f(-3)=(-3)^2-7\cdot (-3) + 4=9+21+4=34.\]

  \[f(2t)=(2t)^2-7\cdot (2t) + 4=4t^2-14t+4.\]
\item
  The domain of a rational functions consists of all real numbers except
  those that make the denominator zero. In this case, the solution
  \(x=\frac54\) of the equation \(5-4x=0\) should be removed. So the
  domain is \(x\neq\frac54\). In interval notation,
  \((-\infty, \frac54)\cup(\frac54,\infty)\).
\item
  One method to solve a linear system is to eliminate one variable first
  using multiplication/division and addition/subtraction. For example,
  to eliminate \(x\), multiplying the first equation by 3 and the second
  the equation by 4, then taking the sum yields
  \(3(-2y)+4(-5y)=3\cdot 16+4\cdot 1\). Simplifying and solving for
  \(y\) implies \[
  \begin{aligned}
  3(-2y)+4(-5y)=&3\cdot 16+4\cdot 1\\
  -26y=&52\\
  y=&-2.
  \end{aligned}
  \]

  To get \(x\), the elimination method can be used again. Or one can
  plug \(y=-2\) into one of the equations and solve for \(x\).

  Plug \(y=-2\) into the first equation and solve for \(x\). \[
  \begin{aligned}
  4x-2(-2)=&16\\
  4x+4=&16\\
  4x=&12\\
  x=&3.
  \end{aligned}
  \]

  So the solution of the system is \((3, -2)\).
\item
  When solving this type of compounded inequality, make sure to apply
  changes to all three places separated by the inequality signs. \[
  \begin{array}{rcccl}
  -12 &\le& 3x-2 &<&7\\
  -10 &\le& 3x &<&9\\
  -\frac{10}{3}&\le& x &<&3.
  \end{array}
  \] In interval notation, the solution set is
  \(\big[-\frac{10}{3}, 3\big)\).
\item
  The slope \(m_{\parallel}\) of the parallel line equals the slope
  \(m\) of the given line. To get the slope of the given line, solve for
  \(y\). \[
  \begin{aligned}
  6x-3y=&12\\
  -3y=&-6x-12\\
  y=&2x+4.
  \end{aligned}
  \] So \(m_\parallel=m=2\). The point-slope form equation of the
  parallel line is

  \[y=2(x-2)-3.\]

  Simplifying the right hand side give the slope-intercept form equation

  \[y=2x-7.\]

  The \(x\)-intercept has the \(y\)-coordinate \(0\). Let \(y=0\) in the
  slope-intercept form and solve for \(x\) yields \(x=\frac72\). So the
  \(x\)-intercept is \(\left(\frac{7}{2},0\right)\).
\item
  Since the unknown \(b\) is in the denominator, to solve for \(b\), we
  may first clearing denominators by multiplying the LCD \(abc\) to both
  sides and then simplify and solve. \[
  \begin{aligned}
  \frac1a=&\frac1b+\frac1c\\
  abc\cdot\frac1a=&abc\cdot\frac1b+abc\cdot\frac1c\\
  bc=&ac+ab\\
  bc-ab=&ac\\
  b(c-a)=&ac\\
  b=&\frac{ac}{c-a}\quad\text{if }a\ne c.
  \end{aligned}
  \] If \(a=c\), there is no solution for \(b\).
\item
  The binomial involves cubes, the formula

  \[A^3\pm B^3=(A\pm B)(A^2\mp AB+B^2)\]

  is needed. \[
  \begin{aligned}
  &8x^6+27y^3\\
  =&(2x^2)^3+(3y)^3\\
  =&(2x^2 + 3y)((2x^2)^2-(2x^2)(3y)+(3y)^2)\\
  =&(2x^2 + 3y)(4x^4-6x^2y+9y^2).
  \end{aligned}
  \]
\item
  A complex rational expression is the quotient of a rational expression
  by another rational expression. One way to simplify it is to simplify
  both the numerator and denominator first and then convert the quotient
  into a multiplication. \[
  \begin{aligned}
  \frac{\frac1x-\frac1y}{1-\frac{x^2}{y^2}}
  =&\frac{\frac{y-x}{xy}}{\frac{y^2-x^2}{y^2}}\\
  =&\frac{y-x}{xy}\cdot\frac{y^2}{y^2-x^2}\\
  =&\frac{y^2(y-x)}{xy(y-x)(y+x)}\\
  =&\frac{y}{x(y+x)}.
  \end{aligned}
  \]
\item
  To evaluate \(f(a)\), substitute \(x\) by \(a\) in the equation.

  \[f(0)=\sqrt[3]{3\cdot 0-8}=\sqrt[3]{-8}=-2.\]

  \[f(24)=\sqrt[3]{3\cdot 24-8}=\sqrt[3]{64}=4.\]
\item
  If the denominator is a binomial involving only square roots,
  multiplying both the numerator and the denominator by the conjugate
  will eliminate the square root signs from the denominator using the
  difference of squares formula \((A-B)(A+B)=A^2-B^2\). \[
  \begin{aligned}
  \frac{6}{3\sqrt{x}-2}=&\frac{6(3\sqrt{x}+2)}{(3\sqrt{x}-2)(3\sqrt{x}+2)}\\
  =&\frac{18\sqrt{x}+12}{(3\sqrt{x})^2-2^2}\\
  =&\frac{18\sqrt{x}+12}{9x-4}.
  \end{aligned}
  \]
\item
  To simplify a radical, the following formula can be used:
  \(\sqrt[n]{x^m}=x^p\sqrt[n]{x^r}\), where \(p\) is the quotient of
  \(m\) by \(n\) and \(r\) is the reminder.

  \[\sqrt{24x^9y^6}=\sqrt{3\cdot 2^3x^9y^6}=2x^4y^3\sqrt{3\cdot 2 x}=2x^4y^3\sqrt{6x}.\]
\item
  Before combining like radicals, the radicals should be simplified
  first. \[
  \begin{aligned}
  2\sqrt{9}-5\sqrt3-\sqrt{75}=&2\cdot 3-5\sqrt3-\sqrt{3\cdot 5^2}\\
  =&6-5\sqrt3-5\sqrt3\\
  =&6-10\sqrt3.
  \end{aligned}
  \]
\item
  To get rid of the radical sign from an isolated radical, one can raise
  both sides to the power that is the index of the radical. \[
  \begin{aligned}
  \sqrt[3]{5+3x}=&-4\\
  (\sqrt[3]{5+3x})^3=&(-4)^3\\
  5+3x=&-64\\
  3x=&-69\\
  x=&-23.
  \end{aligned}
  \]

  \textbf{Check:} Plugging \(x=-23\) into the left hand side implies
  \(\sqrt[3]{5+3\cdot(-23)}=\sqrt[3]{-64}=-4\) which equals the right
  hand side. So \(x=-23\) is the solution of the equation.
\item
  Simplifying a quotient of two complex numbers is similar to
  rationalize a denominator. \[
  \begin{aligned}
  \frac{5\mathbf{i}}{-2+3\mathbf{i}}=&\frac{5\mathbf{i}(-2-3\mathbf{i})}{(-2+3\mathbf{i})(-2-3\mathbf{i})}\\
  =&\frac{-10\mathbf{i}-15\mathbf{i}^2}{(-2)^2-(3\mathbf{i})^2}\\
  =&\frac{-10\mathbf{i}+15}{4+9}\\
  =&\frac{15}{13}-\frac{10}{13}\mathbf{i}.
  \end{aligned}
  \]
\item
  The \(ax^2+bx+c=0\) can be completed into the perfect square form
  \(a(x-h)^2+k=0\), where \(h=-\frac{b}{2a}\) and \(k=ah^2+bh+c\). In
  this case, since \(a=1\) and \(b=-6\), \(h=-\frac{-6}{2\cdot 1}=3\)
  and \(k=3^2-6\cdot 3+3=-6\). Therefore, \[
  \begin{aligned}
  x^2-6x+3=&0\\
  (x-3)^2-6=&0\\
  (x-3)^2=&6\\
  x-3=&\pm\sqrt{6}\\
  x=&3\pm\sqrt{6}.
  \end{aligned}
  \]
\item
  Suppose the longer leg is \(x\) feet, the shorter leg is then \(x-4\)
  feet. By the Pythagorean theorem, that is, in a right triangle,
  \(\text{leg}^2+\text{leg}^2=\text{hypotenuse}^2\). Then the unknown
  \(x\) satisfies the following equation:

  \[x^2+(x-4)^2=7^2.\]

  To get \(x\), simplify and solve the equation. Recall that
  \((A\pm B)^2=A^2\pm 2AB+B^2\) and the quadratic formula
  \(x=\frac{-b\pm\sqrt{b^2-4ac}}{2a}\) if \(ax^2+bx+c=0\). \[
  \begin{aligned}
  x^2+(x-4)^2=&7^2\\
  x^2+x^2-2\cdot 4x+4^2=&49\\
  2x^2-8x+16=&49\\
  2x^2-8x-33=&0\\
  x=&\frac{-(-8)\pm\sqrt{(-8)^2-4\cdot2\cdot(-33)}}{2\cdot 2}\\
  x\approx -2.53\quad\text{or}&\quad x\approx 6.53.
  \end{aligned}
  \] Therefore, the longer leg is approximately \(6.53\) feet and the
  shorter leg is approximately \(6.53-4=2.53\) feet.
\item
  The \(x\)-intercepts have the \(y\)-coordinate \(0\). To get their
  \(x\)-coordinates, solve the equation \(-x^2+4x+5=0\). \[
  \begin{aligned}
  -x^2+4x+5=&0\\
  x^2-4x-5=&0\\
  (x+1)(x-5)=&0\\
  x+1=0\quad\text{or}&\quad x-5=0\\
  x=-1\quad\text{or}&\quad x=5.
  \end{aligned}
  \] So the \(x\)-intercepts are \((-1, 0)\) and \((5, 0)\).

  The \(y\)-intercept is \((0, f(0))=(0, c)=(0, 5)\).

  The equation of the axis of symmetry is \(x=-\frac{b}{2a}\) Since
  \(-\frac{b}{2a}=-\frac{4}{2\cdot(-1)}=2\), the axis of symmetry is
  \(x=2\).

  The vertex is the intersection of the axis of symmetry and the
  parabola. So the coordinates are
  \(\left(-\frac{b}{2a}, f\left(-\frac{b}{2a}\right)\right)\). Since
  \(-\frac{b}{2a}=2\) and \(f(2)=-2^2+4\cdot 2+5=9\), the vertex is
  \((2, 9)\).

  To sketch the graph, plot the intercepts, the vertex, graph the axis
  of symmetry and connect the points smoothly so that the graph is
  symmetric with respect to the axis of symmetry.
\item
  Since \(x=5.5\), the value of \(V\) is \[
  V=35000\cdot (3.21)^{-0.05\cdot 5.5}\approx 25396.81.
  \]
\item
  To solve a logarithmic equation, first combine the logarithmic
  expressions into a single logarithm using rules of logarithms, then
  using the definition of logarithm to rewrite into an equivalent
  equation without logarithms. \[
  \begin{aligned}
  \log_4x+\log_4(x-6)=&2\\
  \log_4[x(x-6)]=&2\\
  x(x-6)=&4^2\\
  x^2-6x=&16\\
  x^2-6x-16=&0\\
  (x+2)(x-8)=&0\\
  x+2=0\quad\text{or}&\quad x-8=0\\
  x=-2\quad\text{or}&\quad x=8.
  \end{aligned}
  \] Since \(\log_4(-2)\) is undefined, only \(x=8\) is a solution.
\item
  The statement that the balance is doubled means \(A=2P\). The years
  \(t\) it takes to double the investment is then satisfies the equation
  \(2P=P\left(1+\frac{r}{n}\right)^{nt}\). Dividing \(P\) from both
  sides yields \(2=\left(1+\frac{r}{n}\right)^{nt}\). Since
  \(r=4.5\%=0.045\), \(n=2\), the equation becomes \[
  \begin{aligned}
  2=&\left(1+\frac{0.045}{2}\right)^{2t}\\
  \log 2=&\log\left(1+\frac{0.045}{2}\right)\cdot (2t)\\
  t=&\frac{\log 2}{2\log\left(1+\frac{0.045}{2}\right)}\\
  t\approx&15.58.
  \end{aligned}
  \] So it takes approximately 15.58 years to double the balance if the
  interest rate is 4.5\% and compounded semiannually.
\item
  When adding/subtracting rational expressions, the rational expressions
  should be rewritten into equivalent expressions with the LCD as the
  denominators. \[
  \begin{aligned}
  &\frac{4x-4}{x^2+2x-15}-\frac{3}{x+5}\\
  =&\frac{4x-4}{(x-3)(x+5)}-\frac{3(x-3)}{(x-3)(x+5)}\\
  =&\frac{(4x-4)-3(x-3)}{(x-3)(x+5)}\\
  =&\frac{4x-4-3x+9}{(x-3)(x+5)}\\
  =&\frac{x+5}{(x-3)(x+5)}\\
  =&\frac{1}{x-3}.
  \end{aligned}
  \]
\item
  The division of rational expression can be rewritten as a product. \[
  \begin{aligned}
  &\frac{x^2-x-12}{2x+8}\div\frac{2x^2+5x-3}{8x-4}\\
  =&\frac{x^2-x-12}{2x+8}\cdot\frac{8x-4}{2x^2+5x-3}\\
  =&\frac{(x+3)(x-4)}{2(x+4)}\cdot\frac{4(2x-1)}{(2x-1)(x+3)}\\
  =&\frac{2(x-4)}{x+4}.
  \end{aligned}
  \]
\item
  In the equation that defines the function, we know that \(R=60000\).
  Then \(x\) satisfies the equation \(60000=x(1000-4x)\). Solving the
  equation yields \[
  \begin{aligned}
  60000=&1000x-4x^2\\
  4x^2-1000x+60000=&0\\
  (4x-600)(x-100)=&0\\
  4x-600=0\quad\text{or}&\quad x-100=0\\
  x=150\quad\text{or}&\quad x=100.
  \end{aligned}
  \] Therefore, when 100 or 150 products were sold, the revenue is
  60000.
\item
  When the value is \(20000\), the age \(x\) satisfies the equation
  \(20000=40000(1.23)^{-0.4x}\). Solving the equation yields \[
  \begin{aligned}
  20000=&40000(1.23)^{-0.4x}\\
  0.5=&(1.23)^{-0.4x}\\
  \log(0.5)=&\log((1.23)^{-0.4x})\\
  \log(0.5)=&\log(1.23)\cdot(-0.4x)\\
  x=&\frac{\log(0.5)}{-0.4\cdot\log(1.23)}\\
  x\approx&8.37.
  \end{aligned}
  \] When the car is 8.37 years old, the value of the car becomes 20000.
\item
  In terms of the variables of the function, that a person weighs 80
  kilogram means \(x=80\). Then the calories the person needed is

  \[f(80)=70\cdot(80)^{3/4}\approx 1872.47.\]

  To mark the point, \((80, f(80))\), draw a vertical line through
  \(80\), the intersection of the vertical line with the graph is the
  point on the graph.

  Suppose the person needed 1500 calories weighs \(x\) kilogram, then
  \(f(x)=1500\), that is \(70\cdot x^{3/4}=1500\). Solving the equation
  yields \[
  \begin{aligned}
  70\cdot x^{3/4}=&1500\\
  x^{3/4}=&\frac{150}{7}\\
  (x^{3/4})^4=&(\frac{150}{7})^4\\
  x^3=&(\frac{150}{7})^4\\
  x=&\sqrt[3]{(\frac{150}{7})^4}\\
  x=&(\frac{150}{7})^{4/3}\\
  x\approx & 59.52\\
  \end{aligned}
  \] A person weighs 59.52 kilogram needs approximately 1500 calories
  per day.
\end{enumerate}



\end{document}
